\documentclass[a4paper,10pt]{report}

\usepackage[utf8]{inputenc}
\usepackage[textheight=730pt, textwidth=480pt]{geometry} % http://ctan.org/pkg/geometry
\usepackage{framed}
%\usepackage{bookman}

%-make the table of content clickable
\usepackage{hyperref}
\hypersetup{
	colorlinks,
	citecolor=black,
	filecolor=black,
	linkcolor=black,
	urlcolor=black
}

%-page style
\usepackage{fancyhdr}
\usepackage{lastpage}

%-include pictures
\usepackage{graphicx}

\setlength{\headheight}{18pt}
\setlength{\footskip}{50pt}

\pagestyle{fancy}
\fancyhf{}
\fancyhead[LE, RO]{\textit{\today}}
\fancyhead[RE, LO]{\large{\textit{\emph{\jobname}}}}

\fancyfoot[LO, RE]{\textit{Page \thepage / \pageref{LastPage}}}


\title{
    \huge CONVENTION FOR THE DEVELOPERS OF CRACKER
    \\
    \LARGE (CDC)
}
%\author{By \emph{Elerias}}
\author{
    \includegraphics[scale=0.55]{../Style/Cracker_ascii_logo.png}
    \\ \\ \\ \\
    \LARGE{\fontfamily{pzc}\selectfont Elerias, Lasercata}
}

\begin{document}
 \maketitle
 \tableofcontents
 \includegraphics[scale=0.55]{../Style/Cracker_ascii_logo.png}
 
 \chapter{Introduction}
  To do.
 
 \chapter{Language Conventions}
  \section{Code}
   The language of the code is English. All the name of variables, functions,
  classes or other objects, the documentation and the description of functions
  and modules should be written in English.
  
  \section{Annotations}
   Annotations can be written in English or in French because all the developers
  speak french, but the English is prefered for a more consistent code.

  
  \section{Communication with user}
   All the outputs in the program should be written in English. The program should take English and French inputs. After a closed question, Cracker should take the following answers :
   \begin{itemize}
   \item[$-$] y ;
   \item[$-$] Y ;
   \item[$-$] yes ;
   \item[$-$] Yes ;
   \item[$-$] YES ;
   \item[$-$] n ;
   \item[$-$] N ;
   \item[$-$] no ;
   \item[$-$] No ;
   \item[$-$] NO ;
   \item[$-$] non ;
   \item[$-$] Non ;
   \item[$-$] NON.
   \end{itemize}
   
   \noindent Examples :
   \par \setlength{\leftskip}{10pt} Text from file ? (y/n) yes
   \par \setlength{\leftskip}{10pt} Text from file ? (y/n) Non
  
  \section{Documents}
   The documents should be written in English. French is however acceptable because all the developers speak french.
   
 \chapter{Style Conventions}
  \section{Names}
   Names of objets have to be short. \\
   In Cracker, some variables have the same functions in different modules. They should have the same name :
   
   \begin{itemize}
   \item ``\texttt{t}", ``\texttt{txt}" or ``\texttt{text}" represents the text that the program use. It is often asked to the user ;
   \item "\texttt{ret}" represents the text returned to the user ;
   \item ``\texttt{t}" or ``\texttt{t\_}" + number represents a time ;
   \item ``\texttt{alf}", ``\texttt{alph}" or ``\texttt{alphabet}" represents an alphabet ;
   \item ``\texttt{f}", ``\texttt{fn}", ``\texttt{filename}" or ``\texttt{f\_name}" represents the name of a file ;
   \item ``\texttt{f\_ext}" represents the extension of the file ;
   \item ``\texttt{w}", ``\texttt{wrd}" or ``\texttt{word}" represents a word ;
   \item ``\texttt{wrdlst}" or ``\texttt{wrdlt}" represent a wordlist ;
   \item ``\texttt{D}", ``\texttt{d}", ``\texttt{dico}" or ``\texttt{dct}" represents a dictionnary ;
   \item ``\texttt{l}", ``\texttt{lth}" or ``\texttt{lenth}" or ``\texttt{length}" represents a length ;
   \item ``\texttt{n}" represents an integer ;
   \item ``\texttt{p}" represents a prime number ;
   \item ``\texttt{d}" or ``\texttt{div}" represents a divisor ;
   \item ``\texttt{L}" or ``\texttt{lst}" represents a list ;
   \item ``\texttt{r}" or ``\texttt{rest}" represents the rest of a euclidean division ;
   \item ``\texttt{M}" or ``\texttt{msg}" represents a plain message ;
   \item ``\texttt{C}" or ``\texttt{msg\_c}" represents a encrypted message ;
   \item ``\texttt{key}" or ``\texttt{k}" represents a key ;
   \item ``\texttt{p}" or ``\texttt{primality}" represents the primality ;
   \item ``\texttt{x}" represents a preimage in a function ;
   \item ``\texttt{y}" represents an image in a function ;
   \item ``\texttt{f}" or ``\texttt{func}" represents a function ;
   \item ``\texttt{window}", ``\texttt{wind}", ``\texttt{win}", or ``\texttt{w}" represent a window (Tkinter or PyQt5).
   
   \end{itemize}

   
  
  \section{Whitespaces}
   Before and after the equal symbol (\texttt{=}), the test symbols (\texttt{'==' '!=' '>' '>=' '<' '<='}) and \texttt{';'}, one whitespace should be placed. \\

   \noindent Good examples :
   
   \texttt{a = 8 ; b = 7}
   
   \texttt{c = 2} \\
   
   \texttt{if a == 8:}
       \par \setlength{\leftskip}{40pt} \noindent \texttt{print(a)} \\
       
   \setlength{\leftskip}{10pt} \noindent Bad examples :
   
   \texttt{a $\quad$ = $\quad$ 8; b= 7}
   
   \texttt{c=2}
   
   \texttt{if a==8 :}
   
   \setlength{\leftskip}{20pt} \texttt{print(a)} \newline
   
   \setlength{\leftskip}{10pt} \noindent Before and after an operator (\texttt{+ - * / // \% **}), one whitespace should but if there is a proritary operation. \\
   
   \noindent Good examples :
   
   \texttt{a = 2 + 4 + 18}
   
   \texttt{b = a*2 + 4 + 18/3}
   
   \texttt{c = a * (b + 3**2)}
   
   \texttt{c = a \% (b + 3**2)} \\
   
   
   \noindent Bad examples :
   
   \texttt{a = 2+4+18}
   
   \texttt{b = a * 2 + 4 + 18 / 3}
   
   \texttt{c = a*(b + 3 ** 2)}
   
   \texttt{c = a \% (b+3 ** 2)} \newline \newline
   
   
   \noindent After '\texttt{:}' and '\texttt{,}', one whitespace should be placed. But not before. \\
   
   \noindent Good examples :
   
   \texttt{if a == 2:}
   
   \indent \indent \texttt{d = \{'a': 14, 23: 'test'\}}
   
   \indent \indent \texttt{L = [1, 'hello', 42]} \\
   
   
   \noindent Bad examples :
   
   \texttt{if a == 2 :}
   
   \indent \indent \texttt{d = \{'a' : 14, 23:'test'\}}
   
   \indent \indent \texttt{L = [1 , 'hello',42]} \newline \newline
   
   
   \noindent For other symbols ( \texttt{( ) \{ \} [ ] ' " .})  , no whitespace should be placed. \\
   
   \noindent Good examples :
   
   \texttt{L = [3, 90, 'bonjour', (4, 2)]}
   
   \texttt{a = 2 * (4+2)}
   
   \texttt{b = f(2)} \\
   
   
   \noindent Bad examples :
   
   \texttt{L = [ 3, 90, 'bonjour', ( 4, 2 ) ]}
   
   \texttt{a = 2 * ( 4+2 )}
   
   \texttt{b = f ( 2 )}
  
  
  
  \section{Indentations}
   Four spaces, automatic indentations and tabulations can be used to indent.
  
  \section{Blank lines}
   Blanks lines improve \emph{readability}, so they should be inserted between the differents \emph{code portions}, i.e. between the import portion, the functions, the classes, the run portion. Blanks lines can also be inserted in the \emph{functions} and \emph{classes}, to separate the differents parts. \\
   
   The number of blanks line to insert depends on the code's length. \emph{Three} blanks lines should be inserted at maximum. \\
   
   One blank line should be inserted after the \emph{documentation} in the classes and functions.
  
  
  \section{Single and double quote}
   For lists, tuples and dictionnaries, single quotes are prefered to lighten the code. If there should be a quote in your string, it is prefered to use the others. Triple quotes are strongly discouraged. \\
  
  
   \noindent Good examples :
   
   \texttt{d = \{`b': 8, `23': `hello', `logo' : "Cracker's logo"\}}
   
   \texttt{L = [7, `afternoon', (`monday', `wednesday'), `"quoted"']} \\
   
   
   \noindent Bad examples :
   
   \texttt{d = \{"b": 8, "23": "hello"\}}
   
   \texttt{L = [7, "test", "$\backslash$"quoted$\backslash$"", `Cracker$\backslash$'s logo']} \\
   
   
   \noindent Very bad example :
   
   \texttt{L = [7, ```afternoon''', ("""monday""", """wednesday""")]} \newline \newline
   
   
   \noindent To frame a word, single quotes are also recommended. \\
   
   \noindent Good examples :
   
   \texttt{a = `hello'}
   
   \texttt{b = "Cracker's logo"}
   
   \texttt{c = `"This" is quoted'} \\
   
   
   \noindent Bad examples :
   
   \texttt{a = """hello"""}
   
   \texttt{b = `Cracker$\backslash$'s logo'}
   
   \texttt{c = "$\backslash$"This$\backslash$" is quoted"} \newline \newline
   
   
   \noindent To define an empty character string, 2 double quotes are prefered. \\
   
   \noindent Good example :
   
   \texttt{a = ""}
   
   \noindent Rather not :
   
   \texttt{a = `'}
   
   \noindent Absolutely not :
   
   \texttt{a = """"""}
  
  
  
  \section{Docstrings}
   Both single and double quoted are accepted in the docstrings, but only one type of quote should be used within a same file, to have a consistent code.
   
   If the description line measures one line, the 2 docstrings should be on the same line, else they have to be on the same column. \\
   
   \noindent Good examples :
   
   \texttt{def f(x):}
   
   \indent \indent \texttt{"""Return the square of x."""} \\
   
   \indent \indent \texttt{return x ** 2} \\
   
   \texttt{def g(x):}
   
   \indent \indent \texttt{```Return the inverse of x. \newline
	   \indent \indent x : a number different of 0. \newline
	   \indent \indent''' \newline \newline
	   \indent \indent return 1 / x} \\
   
   \texttt{def h(x, n): \newline
	   \indent \indent """\newline 
	   \indent \indent Return the \emph{n}th root of x. \newline \newline
	   \indent \indent x : a float number ; \newline
	   \indent \indent n : an other float number. \newline
	   \indent \indent """ \newline \newline
	   \indent \indent return x **(1 / n)} \newline \newline
   
   
   \noindent Bad examples :
   
   \texttt{def f(x): \newline
	   \indent \indent """Return the square of x. \newline
	   \indent \indent """ \newline \newline
	   \indent \indent return x ** 2} \\
  
   \texttt{def g(x): \newline
	   \indent \indent """Return the inverse of x. \newline
	   \indent \indent x : a number different of 0""" \newline \newline
	   \indent \indent return 1 / x} \\
   
   \texttt{def h(x, n): \newline
	   \indent \indent """\newline 
	   \indent \indent Return the \emph{n}th root of x. \newline \newline
	   \indent \indent x : a float number ; \newline
	   \indent \indent n : an other float number.""" \newline \newline
	   \indent \indent return x **(1 / n)}
  
  
  \section{Annotations}
   Todo.
  
  \section{Imports}
   Todo.
  
  \section{Modules}
   Todo.
  
  
  
 \chapter{Using of Craker functions}
  Todo
  
  
  
 \chapter{Documents about Cracker}
  \section{Updates}
   An update note should be fill every time the program is modified. It should be saved in the folder \texttt{Cracker\_v[version]/updates/updates\_notes}, and it should be named ``\texttt{update\_YYYY-MM-DD\_[new version].txt}'' (with the brackets, this time). \\
   
   The update note should begin by the date at the format ``\texttt{YYYY.MM.DD}'' (or can be ``\texttt{YYYY.MM.[day begin]-[day finish]}'' (example : \texttt{2020.05.01-09}) if the update took more than one day, but it is not needed), and should be followed, the line after, by ``\texttt{Cracker v[new version] <-- v[old version]}''. There should be a part for every module improved, begining by ``\texttt{[module name] v[new version] <-- v[old version] :}'', and ending by ``-''$\times 3$. The first part should be consecreted to the general improvements, begin by ``\texttt{General improvements (from [old Cracker version]) :}'', and end by ``-''$\times 6$.
   
   \noindent The update note should end by a separation followed the line after by ``\texttt{By [author]}'' \\
   
   \noindent The update notes should be written in raw text, with the ``\texttt{.txt}'' extension. \newline \newline
   
   
   
   \noindent Example :
   
   file name : ``\texttt{update\_2020-05-02\_[2.4.0].txt}'' \\
   
   \begin{framed}
    \texttt{2020.05.02 \newline
	   \indent Cracker v2.4.0 <-- v2.3.2 \newline
	   \indent --------------------------------- \newline
	   \indent General improvements (from 2.3.2) : \newline
	   \indent \indent - Small bug corrections ; \newline
	   \indent \indent - Improving Crypta and Prima. \newline
	   \indent ------ \newline \newline
	   \indent prima v3.0 <-- v2.4 : \newline
	   \indent \indent - New algorithms ; \newline
	   \indent \indent - Pollard's rho algorithm is the fastest. \newline
	   \indent --- \newline \newline
	   \indent crypta v2.9 <-- v2.8 : \newline
	   \indent \indent - Joining AES in the menu. \newline
	   \indent --- \newline \newline
	   \indent --------------------------------- \newline
	   \indent By Elerias}
   \end{framed}

  
  \section{History}
   The file ``\texttt{history.txt}'' contain all the majors improvements of \emph{Cracker}. It is situated at \texttt{Cracker\_v[version]/updates}. If this software is improved enough, i.e. if the version pass from \texttt{x.y.z} to \texttt{x+1.0.0} or \texttt{x.y+1.0} (First or second digit incremented), the file should be updated. To update this file, the \texttt{General improvements} part of the update note should be added to the \texttt{history.txt} file.
  
  \section{Version}
   The file ``\texttt{Version.txt}'' contain the actual \emph{Cracker} version. The version should contain \textbf{three} numbers, separeted by two dots, i.e. at the format \texttt{x.y.z}, where \texttt{x}, \texttt{y}, and \texttt{z} are positive integers. \newline
   
   \noindent Example : \texttt{2.4.0} \\
   
   If the actual version is \texttt{x.y.9}, the next version should not be \texttt{x.y+1.0}, but \underline{\texttt{x.y.10}}. The same rule applies to the others digits. \\
   
   If only some small bugs were fixed, whitout major improvements, only the \emph{last digit} (\texttt{z}) should be incremented. If some improvements were added, and maybe new modules added, the \emph{second digit} (\texttt{y}) should be incremented, and the last should be set to 0. If the program was improved generaly, maybe with a total re-organisation, a lot of functionnal upturns, and a lot of modules improved, then the \emph{first digit} (\texttt{x}) should be incremented. \newline \newline
   
   
   \noindent Good exemples :
   
   \texttt{2.3.9} $\rightarrow$ \texttt{2.3.10} (small bux corrections)
   
   \texttt{2.4.3} $\rightarrow$ \texttt{2.5.0} (new module, launchers improved)
   
   \texttt{1.5.2} $\rightarrow$ \texttt{2.0.0} (rewriting some parts, put the project in modules) \\
   
   
   \noindent Bad examples :
   
   \texttt{2.3.9} $\rightarrow$ \texttt{2.4.0} (small bux corrections)
   
   \texttt{2.4.3} $\rightarrow$ \texttt{2.5.3} (new module)
   
   \texttt{1.5.2} $\rightarrow$ \texttt{2.5.0} (bug fix)
  
   
  \section{Version\_modules}
   The file ``\texttt{Versions\_modules.txt}'' contain the actual \emph{Cracker}'s modules versions. They should correspond to the variables set at the top of the modules (\texttt{[module name]\_\_ver}). They should be composed of \textbf{two} numbers, separated by one dot. If the version need to contain more than two numbers, separate the lasts with a comma (\texttt{,}). \\
   
   \noindent Exampe :
    \begin{framed}
     \texttt{
		 \noindent Version of cracker\_launcher : 1.0,1 \newline
		 \indent Version of cracker\_gui\_launcher : 1.1 \newline
		 \indent Version of cracker\_parser : 2.0 \newline \newline
		 \indent Version of cracker\_console\_functions : 1.1 \newline
		 \indent Version of base\_functions : 2.2 \newline
		 \indent Version of color : 2.2 \newline
		 \indent Version of matrix : 1.1 \newline \newline
		 \indent Version of Hasher : 3.3 \newline
		 \indent Version of hash\_crack\_2 : 1.0 \newline
		 \indent Version of wordlist\_generator : 6.4 \newline
		 \indent Version of Crypta : 2.9 \newline
		 \indent Version of Prima : 3.0 \newline
		 \indent Version of Base convert : 2.1,2 \newline
		 \indent Version of P@ssw0rd\_Test0r : 1.0}
    \end{framed}
    
    \noindent When a module is created, its version should be \texttt{1.0}.

  
  \section{CDC}
   Todo
 
\end{document}
